\documentclass{subfile}

\begin{document}
An ``artificial neural network" (ANN) is a construct for fitting a function to data (i.e. using empirical risk minimization) consisting of a set of ``composed"
basis function methods which are fit simultaneously.

Suppose we have some data $X$ and $y$, in general arbitrary rank tensors.  The goal is to learn a function $F: X\to \hat{y}$ which approximates the conditional
probability distribution $P(y|X)$.  At its simplest,
deep learning takes $F$ to be of the form
\begin{align}
\begin{split}
    F(X) = \hat{y} = f_{(n)}(W^{(n)}\cdot y^{(n-1)}+b^{(n)}) \\
    y^{(n-1)} = f_{(n-1)}(W^{(n-1)}\cdot y^{(n-2)}+b^{(n-1)}) \\
    \cdots \\
    y^{(1)} = f_{(1)}(W^{(1)}\cdot X+b^{(1)})
\end{split}
\end{align}
where the $f_{(n)}$ are (typically non-linear) element-wise functions.  The $W^{(n)}$ and $b^{(n)}$ are referred to as weight and bias tensors, respectively,
their rank and dimensions are determined by $X$ and depend on the implementation.  They are typically randomly initialized according to a simple distribution,
which we will address in a future section.  Note that also more elaborate function topologies (ways of inputting the $y^{(n)}$'s to different basis functions)
are possible and even common.  We will discuss these in future sections but they are hard to express in the general case.  Note that typically the $f_{(n)}$
have no parameters of their own, so that the $W$'s and the $b$'s constitute all of the parameters of the model.  Since these are typically tensors of rank 2 or
higher, it is common to have ANN's with millions of parameters.  It is standard practice not to concern oneself with the number of parameters in the model, and
one may even choose underdetermined models.  This is essentially the key novelty of ``machine learning" (ML) which we will briefly discuss momentarily.

The loss function depends on the implementation, but in most cases these are something simple like $L_{2}$ (in the continuous case) or cross-entropy (in the
discrete case).  Optimization is usually performed using some variant of stochastic gradient descent SGD).  Of course symbolically computing the gradient
$\nabla_{\theta}F(X;\theta)$ is trivial using the chain rule, but one must be careful of the computational implementation since a naive approach can lead to a
combinatoric explosion of terms and hence time complexity.

\subsection{Novelties of ML}
The principle innovation of ML is that attempting to optimze loss functions following from models like those introduced above using stochastic gradient descent
often leads to reasonable approximations of ``real-world" $P(X,y)$.  Understanding why this should be the case would require understanding the underlying
structure of the distributions being fit.  This is a daunting task since ML is effective for a wide variety of different regression and classification tasks.  A
small minority of the literature pursues this question \cite{renorm,deepcheap}.

Even from an optimistic perspective one might expect the gradient descent to become stuck in one of a huge number of (spurious) local minima.  There are a
number of reasons why this problem is not prohibitive in practice.  For one, the parameter space of $F(X;\theta)$ is typically of very high dimensionality
($\gtrsim 10^{6}$) so the vast majority of critical points are saddle points, where of course SGD does not get stuck.  The second, and much more surprising
reason is that typically the structure of the loss function is such that many of the local minima in the space actually correspond to excellent generalizations
of the data (i.e. functions which are a good fit to $P(X,y)$.  This is quite surprising, and indeed nobody currently understands why this should be the case in
such a wide variety of different applications.

It also bears mentioning that the presence of so many \emph{desirable} local minima (typically found in the vicinity of the true minimum) makes it possible to
train underdetermined models without severely overfitting them.  This may seem remarkable a priori, but follows naturally from the above surprising facts.  Note
however that even though this is the case, often methods of preventing overfitting are employed (even when this involves simply discriminating between local
minima) and will be discussed in a subsequent section.

\end{document}
